\section{Lehramtsstudium allgemein}
\label{lehramt_allg}
Allgemein gilt zu sagen: Ein Lehramtsstudium bedeutet viel mehr organisatorisches Geschick als ein gewöhnlicher Bachelorstudiengang, denn ihr müsst zwei Fächer unter einen Hut bekommen. Zwar gibt es hierzu oftmals sogenannte „Musterstudienpläne“, dennoch sind diese mit Vorsicht zu genießen. Oft überschneiden sich Veranstaltungen und keiner kann genau sagen, welche der Veranstaltungen man nun zuerst hören sollte. Prinzipiell empfehlenswert: Versucht, euch Schwerpunkte zu setzen und konzentriert euch lieber in den ersten Semestern stärker auf eines eurer Fächer, dafür aber richtig, statt zu viele Veranstaltungen nur halbherzig zu besuchen -- auch wenn das den Modellstudienplänen nicht immer entspricht. Diese sind leider nicht immer ideal untereinander abgestimmt, sondern meist nur für das jeweilige Einzelfach konzipiert.


\subsection{Immatrikulation für das Lehramtsstudium}

Grundsätzlich immatrikuliert man sich im Staatsexamensstudiengang Lehramt immer in mindestens zwei Hauptfächern. Gegebenenfalls kann nach der Zwischenprüfung in den Hauptfächern noch ein drittes Fach -- ein sogenanntes Erweiterungsfach -- hinzukommen.

Aber Vorsicht! Manche Fächer können nicht auf Lehramt studiert werden, andere Fächer nur in bestimmten Kombinationen. Obwohl das Studentensekretariat darauf achten sollte, kommt es immer wieder vor, dass Leute mit „falschen“ Kombinationen immatrikuliert werden. Seht auf jeden Fall nochmal selber nach: Welche Fächerkombinationen in Baden-Württemberg auf Lehramt studiert werden können, entnehmt ihr einer Tabelle des Kultusministeriums, die ihr bei verschiedenen Beratungsstellen (meist auch online auf deren Webseite) einsehen könnt: Es gibt für jedes Fach eine eigene Lehramtsberatung sowie eine zentrale Beratung für Lehramtsstudierende durch das Oberschulamt; für rechtsverbindliche Auskünfte sollte man sich an letztere wenden.

\emph{Achtung:} In anderen Bundesländern gelten andere Regelungen!


Zur Immatrikulation für einen Lehramtsstudiengang muss eine Bescheinigung über die Ableistung eines sogenannten Orientierungspraktikums vorgelegt werden. Sofern diese noch nicht vorliegt, kann sie aber für eine begrenzte Zeit noch nachgereicht werden. Die konkrete Frist orientiert sich meist an den sonstigen Fristen der Immatrikulation, lest hierzu am besten auf der Seite des Studierendensekretariats nach oder ruft dort an (06221 -- 54\,54\,54).

\vspace*{-2\parskip}

\subsection{Modularisierung -- jetzt auch für Lehrämtler}

Für alle Studienanfänger ab dem WS10/11 (also insbesondere für euch) gilt die
neue
\gls{GymPO}\footnote{\url{
http://www.landesrecht-bw.de/jportal/?quelle=jlink&query=GymLehrPr1V+BW&psml=bsbawueprod.psml&max=true&aiz=true}}. Zwar wurde in Baden-Württemberg das Lehramtsstudium nicht auf
ein Bachelor/Master-System umgestellt, die Studiengänge wurden allerdings -- aus
Kompatibilitätsgründen mit BA/MA -- modularisiert, d.h. der Lehramtsstudiengang besteht aus Modulen
mit Leistungspunkten (LP), die wie im Bachelor/Masterstudiengang gesammelt und bestanden werden
müssen. Hinzu kommen allerdings im Unterschied zu den BA/MA-Studiengängen ein verbindliches
Praxissemester sowie eine staatliche Abschlussprüfung. Gemeinsam mit den Noten aus den bestandenen
Modulen errechnet sich hieraus dann eure Endnote.

Was die Wiederholungsmodalitäten betrifft, gelten für das Lehramt die Vorgaben des
Bachelorstudiengangs analog,
d.h. nicht bestandene Modulprüfungen können einmal  wiederholt  werden, und zwar zum nächstmöglichen
Prüfungstermin. Wird ein Modul aus dem Pflichtbereich auch im zweiten Versuch nicht bestanden
(„endgültig nicht bestanden“), führt dies zum Verlust des Prüfungsanspruches. Da ihr aber
pro Modul jeweils zwei Prüfungsmöglichkeiten habt (besser bekannt als Freischuss oder Nachklausur),
ist diese Hürde in der Regel unproblematisch.

\subsection{Praxissemester}

Gemäß der Lehramtsprüfungsordnung müssen Lehramtsstudierende ein Schulpraxissemester absolvieren oder eine vergleichbare Schulpraxis (z.B. Assistant Teacher im Ausland) nachweisen.  Das Praxissemester soll -- so das Kultusministerium -- schon früh den Bezug zur Schulpraxis herstellen. (Dass für die Einführung aber auch Kostengründe gesprochen haben, ist kaum zu leugnen. Schließlich wird das Schulpraxissemester nicht bezahlt, im Gegensatz zu dem halben Jahr Referendariat, welches dadurch ersetzt wird.)  Das Praxissemester soll in der Regel nach dem dritten oder vierten Studiensemester, also gegen oder nach Ende des Grundstudiums absolviert werden. Empfehlenswert ist häufig das fünfte Fachsemester, wie es auch in den Studienordnungen der meisten Fächer vorgeschlagen wird. Letztlich könnt ihr aber den Termin frei wählen. Das Praxissemester dauert 13 Wochen – es beginnt zum Schuljahresbeginn im September und endet mit Beginn der Weihnachtsferien.
Während des Praxissemesters besucht man nachmittags Kurse beim Staatlichen Seminar für Didaktik und Lehrerbildung, wo man etwas über Pädagogik und Fachdidaktik lernen soll.

Ihr solltet bereits ein einführenden Vorlesung in Bildungswissenschaft sowie pro Fach eine Fachdidaktikveranstaltung besucht haben, bevor ihr in der Regel im 5. Semester ins Schulpraxissemester geht. Im Frühjahr, bis das kommende Semester startet, habt ihr eventuell die Möglichkeit Blockveranstaltungen zu belegen.

Die Vergabe der Praktikumsplätze wird über das Internet geregelt -- nähere Infos dazu findet man auf der Homepage der jeweiligen Fakultät.

\subsection{Das Begleitstudium}
\label{epg}

Neben eurem Fachstudium in den beiden Fächern müsst ihr noch ein Begleitstudium absolvieren, welches sich formal in drei Teile gliedert: das bildungswissenschaftliche Begleitstudium und das \gls{EPG}. Dieser Teil des Studiums ist mit der neuen Prüfungsordnung deutlich ausgeweitet worden. In Bildungswissenschaft müssen 18 Leistungspunkte erworben werden. Diese erwirbt man in der Regel durch den Besuch der Vorlesungen „Einführung in die Schulpädagogik“ und „Einführung in die Pädagogische Psychologie“ sowie durch den Besuch zweier Seminare am Institut für Bildungswissenschaft. Im EPG müssen insgesamt 12 Leistungspunkte erworben werden. Hier besucht man in der Regel zu gleichen Teilen Veranstaltungen zu Grundlegenden Fragen der Ethik (\gls{EPG} 1) und Veranstaltungen zu fach- oder berufsbezogenen ethischen Fragestellungen (nicht notwendigerweise im eigenen Fach!) (\gls{EPG} 2). Die Note der Seminare geht jeweils schon zu einem kleinen Teil in eure Staatsexamensnote ein.

Mit der \gls{GymPO} völlig neu eingeführt wurde das \glqq Modul personale Kompetenzen\grqq. Viele Bildungswissenschaftliche Seminare lassen sich aber sowohl im Bereich personale Kompetenz, als auch im bildungswissenschaftlichen Begleitstudium anrechnen.

\subsection{Fachdidaktik}

In euer Fachstudium integriert ist eine fachdidaktische Ausbildung. Details hierzu findet ihr in den entsprechenden Artikeln zu euren Hauptfächern.


\subsection{Übersicht}

Hier habt ihr nochmal eine knappe Übersicht darüber, was die \gls{GymPO} für das Lehramtsstudium vorgibt:

\begin{table*}[htb]
\centering

\begin{tabular}{ll}
  \toprule
  Bereich  & Leistungspunkte\\
  \midrule
  1. Hauptfach, Pflichtmodule & 80\\
  1. Hauptfach, Wahlmodule & 14\\
  1. Hauptfach, mündliche Abschlussprüfung & 10\\
  \addlinespace
  2. Hauptfach, Pflichtmodule & 80\\
  2. Hauptfach, Wahlmodule & 14\\
  2. Hauptfach, mündliche Abschlussprüfung & 10\\
  \addlinespace
  Wissenschaftliche Arbeit (in einem der Hauptfächer) & 20\\
  \addlinespace
  \gls{EPG} 1 & \phantom{0}6\\
  \gls{EPG} 2 & \phantom{0}6\\
  Bildungswissenschaftliches Begleitstudium & 18\\
  Personale Kompetenz & \phantom{0}6\\
  Schulpraxissemester & 16\\
  \bottomrule
\end{tabular}

\end{table*}

