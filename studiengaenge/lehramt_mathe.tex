\section{50\%-Bachelor Mathematik (Lehramt)}

Das Studium beginnt mit einem Zweifach-Bachelor in Mathematik und im anderen
gewählten Fach. Der Schwerpunkt liegt in den ersten sechs Semestern also auf
dem fachlichen Studium. In der Mathematik startet man mit den Grundvorlesungen
Ana I und II und LA I und II; jede dieser vier Vorlesungen gibt jeweils 8~LP.

Je nach der Wahl des zweiten Fachs können die ersten beiden Semester damit sehr
voll sein. Viele Lehramtsstudierende konzentrieren sich deshalb entweder auf
Mathematik und hören nur wenige Vorlesungen aus dem zweiten Fach oder
entscheiden sich für eines der beiden Module, also entweder Analysis oder
Lineare Algebra.

Im Zweifach-Bachelor gibt es da leider noch keine Erfahrungen. Solltet ihr
aber feststellen, dass euch vier Vorlesungen im ersten Semester zu viel sind,
dann ist das kein Weltuntergang. Ihr solltet euch dann entweder auf Mathematik
oder auf eines der beiden Module konzentrieren. Das kann natürlich dazu
führen, dass sich euer Studium um ein oder zwei Semester verlängert. Aber in
anderen Bachelor-Studiengängen, zum Beispiel im Mathematik Bachelor 100\%, ist
die durchschnittliche Studienzeit höher als die Regelstudienzeit.
Wenn ihr länger als die Regelstudienzeit studiert, kann es sein, dass ihr für
die zusätzlichen Semester kein BaFög mehr bekommt. Aber auch da gibt es
Ausnahmen. Ihr solltet euch, falls ihr Vorlesungen schiebt, frühzeitig bei den
zuständigen Stellen informieren.

Nachdem ihr die Grundvorlesungen absolviert habt, könnt ihr euch aussuchen, in
welcher Reihenfolge ihr die Vorlesungen aus dem Wahlpflichtbereich hören
wollt. Das sind
\begin{itemize}
  \item Algebra 1,
  \item Funktionentheorie 1,
  \item Einführung in die Wahrscheinlichkeitstheorie und Statistik und
  \item Einführung in die Numerik,
\end{itemize}
alle wieder jeweils zu 8~LP. Im Bachelor müsst ihr drei der vier zur Wahl
stehenden Vorlesungen hören. Eine vierte dürft ihr frei aus dem Angebot der
Fakultät wählen. Außerdem müsst ihr im Bachelor noch jeweils ein Proseminar und
ein Seminar machen; die bringen jeweils 6~LP.

Dabei müsst ihr beachten, dass ihr für den Master of Education alle vier
Vorlesungen bestanden haben müsst. Dazu kommt noch Einführung in die Geometrie,
die ihr nur im Master hören könnt. Ihr könnt aber eine der Vorlesungen erst im
Master hören. Wenn ihr alle vier Vorlesungen schon im Bachelor macht, dann
könnt ihr im Master eine Vorlesung frei aus dem Angebot der Fakultät
auswählen.

Als fachübergreifenden Kompetenzen (FÜK) solltet ihr bildungswissenschaftliche
Veranstaltungen belegen, da diese Zulassungsvoraussetzung für den Master of
Education sind. Welche genau das sind findet ihr im Artikel über das allgemeine
Lehramt. Damit bleiben euch keine Leistungspunkte im Bereich der FÜK mehr übrig.

Als letztes bleibt noch die Entscheidung zu fällen, in welchem Fach ihr eure
Bachelor-Arbeit schreiben wollt. Diese Entscheidung bestimmt, welches euer
erstes Hauptfach ist und ob ihr dann einen Bachelor of Science oder einen
Bachelor of Arts habt.

Solltet ihr nach dem Doppelbachelor doch den Fach-Master machen wollen, ist in
den meisten Fächern dann auch die Bachelorarbeit in diesem Fach Voraussetzung
für den Fach-Master.

Andernfalls folgt der Master of Education. Hier liegt der Schwerpunkt auf der
Didaktik und auf den Bildungswissenschaften. Außerdem macht ihr im zweiten
Master-Semester euer Schulpraxissemester. Mehr Details wissen wir leider noch
nicht, da es noch keine Prüfungsordnung für den Master of Education gibt.
