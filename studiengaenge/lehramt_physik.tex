%\newpage\Large\mathphyssubsubsec{Lehramt Physik}\small
\section{Lehramt Physik}

Zum Wintersemester 2010 wurde eine neue Prüfungsordnung für das Gymnasiallehramt verabschiedet, nach der du als angehende\_r Lehrer\_in dein Studium auszurichten hast. Seit dem hat die Studienkommission (also auch die Fachschaft) viel Zeit und Mühe in die Gestaltung des Lehramtsstudiums gesteckt, um der neuen Prüfungsordnung gerecht zu werden und gleichzeitig die Studierbarkeit des Studiengangs zu erhalten. Das war nicht immer ganz einfach und ist leider auch nicht flächendeckend gelungen. Daher möchten wir dir sehr ans Herz legen, die Mathematik als zweites Fach neben der Physik zu wählen, denn formal werden die für Physik erforderlichen Mathematikkenntnisse zwar in den Theorievorlesungen der Physik abgedeckt, es hängt aber sehr von dem/der Prof und deiner Motivation zum Selbststudium ab, ob das gelingt und ausreicht. Falls du dennoch ein anderes Fach als zweites Fach gewählt hast, könntest du durchaus überlegen, die eine oder andere Grundvorlesung der Mathematik (Analysis 1, Lineare Algebra 1, \dots) freiwillig zu hören. Du solltest aber das Arbeitspensum vor allem in den ersten Semestern nicht unterschätzen und gegebenenfalls die zusätzlichen Vorlesungen wieder fallen lassen.


In der Physik existieren zusätzlich zu den Grundvorlesungen aus dem Bachelor-Studium, auf denen auch euer Studium aufbaut, spezielle Veranstaltungen für euch als Lehramtsstudierende. So gibt es z.B. spezielle Lehrveranstaltungen für das Lehramt in der theoretischen Physik, den physikalischen Praktika und auch bei den Seminaren. Das soll euch aber nicht davon abhalten, euch fröhlich unter die Bachelor zu mischen und auch Veranstaltungen zu besuchen, die nicht explizit für das Lehramt ausgeschrieben sind.

Eigentlich sind alle wichtigen Fakten in Modulhandbuch und der Prüfungsordnung beschrieben, aber einige wesentliche Aspekte seien hier dennoch erläutert. Wenn du Mathematik als zweites Hauptfach gewählt hast, musst du dich im ersten Semester entscheiden, ob du den Schwerpunkt zu Beginn lieber auf die Mathematik oder die Physik legen möchtest. Wir empfehlen dir wärmstens, dich für die Mathematik zu entscheiden, da diese das Handwerkszeug des Physikers bildet -- und ohne Werkzeug arbeitet's sich nur mühsam. Konkret hieße das dann, dass du die Vorlesungen Analysis 1, Lineare Algebra 1 und Experimentalphysik 1 hörst. Im zweiten Semester stehen dann die jeweiligen Vorlesungen mit einer "`2"' auf dem Plan. Im September wirst du nun das erste Praktikum absolvieren, in dem du lernst, wie man Versuche durchführt und auswertet. Im dritten Semester kannst du dann schon eine Vorlesung aus dem Bereich der angewandten Mathematik auswählen und während dir die Experimentalphysik weiterhin erhalten bleibt, hörst du nun zusätzlich die Theoretische Physik 1. So setzt sich das fort, so dass du am Ende des vierten Semesters genug gelernt hast um deine erste große Prüfung abzulegen: die Zwischenprüfung. Sie ist Voraussetzung für den Übergang ins Hauptstudium und muss spätestens zu Beginn des siebten Semesters bestanden sein. In der Regel bestreitest du dann im fünften Semester dein Praxissemester, das heißt du gehst für 13 Wochen in eine Schule und unterrichtest ein bisschen. Du kannst auch erst im siebten Semester das Praxissemester absolvieren, dann musst du die Vorlesungen, die eigentlich dort vorgesehen sind, im fünften Semester hören.

Wenn du bis dahin gekommen bist, blickst du im Uni-Dschungel selbst so weit durch, dass du weißt, wie dein Studium weitergehen wird. Im Regelfall kannst du dann im zehnten Semester deine wissenschaftliche Abschlussarbeit schreiben und das erste Staatsexamen ablegen. Anschließend geht's wieder dahin zurück, wo du gerade herkommst: in die Schule -- und dann kannst du all das, was dich schon immer an deinen Lehrern gestört hat, besser machen.
