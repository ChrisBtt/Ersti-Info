\section{Lehrveranstaltungen an der Uni}
Im Gegensatz zur Schule gibt es an der Uni eine Reihe verschiedenartiger
Veranstaltungen, in denen der Lehrstoff vermittelt werden soll. Im wesentlichen
sind das: Vorlesungen, Übungen, Seminare und Praktika. Ihr habt im ersten
Semester nur mit den ersten beiden zu tun. Diejenigen unter Euch, die Physik
studieren, werden ab dem zweiten Semester auch Praktika machen. Proseminare
werden in Mathe ab dem zweiten Semester angeboten. Beides braucht Euch also
jetzt noch nicht zu beunruhigen.

Theoretisch sollte in den Vorlesungen eine Dozentin oder ein Dozent ein abgestecktes Teilgebiet der Mathematik bzw. Physik vermitteln.

Die Praxis sieht jedoch oft anders aus. In der Physik werden in atemberaubenden
Tempo erstaunliche Phänomene vorgeführt und zur Erklärung noch erstaunlichere
Formeln herangezogen, oft unterstützt durch erstaunlichste Versuche (die
mitunter zur Multi-Media-Show geraten).

In Mathe bedecken die Profs in ebenfalls atemberaubendem Tempo gleichmäßig die
Tafel mit ebenfalls erstaunlichen Zeichen die, kaum definiert, bereits in
abenteuerliche Beweise verwickelt werden.  Diese wiederum ermuntern dann einige
unterforderte Studis dazu, Quervergleiche zum Lebesgueschen Lemma und ebenfalls
verwirrenden Korollaren vorzuschlagen.

Es gibt zwei Wege, um trotz oft unbefriedigender Vorlesungspraxis an Lehrstoff
zu kommen, d.h.  etwas zu verstehen und nicht gleich am Anfang den Faden zu
verlieren. Der eine ist, Fragen zu stellen, auch wenn das häufig Überwindung
kostet. Es ist dabei gar nicht so wichtig, ob mit einer Frage etwas sofort klar
wird, es ist schon gut, dass durch eine Frage den anderen und sich selbst
Gelegenheit gegeben wird, die eigenen Gedanken zu ordnen (und nicht nur
mitschreiben zu müssen).  Außerdem wird nur so den DozentInnen gezeigt, dass es
überhaupt Fragen gibt und nicht alles selbstverständlich ist. Dabei ist zu
sagen, dass man sich mit Fragen nicht unbedingt eine Blöße gibt, denn manche
scheinbar „blöde“ Frage hat schon viele DozentInnen aus dem Konzept geworfen.

Die andere Methode folgt dem Motto: „Gemeinsam macht stark“ . Wenn ihr Euch in
Arbeitsgruppen zusammensetzt, könnt ihr Vieles klären, was Euch alleine völlig
unerklärlich schien. Für das Lösen der Übungsaufgaben (siehe unten) ist es
sowieso unerlässlich zusammenzuarbeiten, denn nur so könnt ihr damit klar
kommen.

Mit dem Wort „Übungen“ wird eigentlich zweierlei bezeichnet: Einerseits die
wöchentlich ausgegebenen Übungsaufgaben, die ihr für Eure Scheine braucht
(siehe dort), andererseits die Übungsgruppen, in denen die Aufgaben besprochen
und Probleme aus den Vorlesungen geklärt werden sollen. Auch hier gibt es
wieder Unterschiede zwischen Mathe/Informatik und Physik: In Physik werden die
Übungen in der Regel mindestens von Promovierten, oft auch von Habilitierten
(also Profs) gehalten, die jedoch der Studiertätigkeit ziemlich entwachsen
sind. Dementsprechend akademisch geht es häufig zu, viele halten eigene
Privatvorlesungen. Und auch hier gilt es, so viele Fragen wie möglich zu
stellen. Im ersten Semester werdet ihr neben dem fakultativen Basiskurs je zwei
Semesterwochenstunden Übungen in Experimentalphysik und Theoretischer Physik
haben. Manchmal werden vor Klausuren auch noch Sonderstunden angeboten, was ihr
allerdings mit Eurem Tutor ausmachen müsst.

In Mathe und Informatik werden die Übungsgruppen von Studierenden höherer
Semester, sogenannten Hiwis, gehalten. In den Übungsgruppen ist es am
leichtesten, Probleme und Verständnisfragen zu klären. Leider werden aber oft
nur die Aufgaben „heruntergerechnet“, man passt nicht mehr auf und versteht
trotzdem nicht mehr von der Aufgabe als vorher. Sollte dies der Fall sein,
fragt penetrant nach dem Sinn der Aufgabenstellung, nach ihrem Zusammenhang mit
der Vorlesung oder was Euch sonst noch durch den Kopf geht. Nur so habt ihr
eine Chance, dass ihr auch etwas aus der nervenzermürbenden Rechnerei lernt. Es
kann z.B. ein Überblick gegeben werden, wozu man einen Satz später braucht oder
auch mal ein Satz aus der Vorlesung ausführlich erklärt werden. Manchmal wird
auch zum Beginn jeder Übung die letzte Vorlesung von einem der Teilnehmer
zusammengefasst. Das gibt demjenigen die Gelegenheit, die Vorlesung intensiver
nachzubereiten, und den anderen, nochmal eine einfachere Darstellung des
Stoffes zu bekommen - und zwar von jemandem, der die gleichen Probleme wie man
selbst hat und eine andere Sichtweise als die Profs.

Natürlich ist es auch möglich, alle Veranstaltungen (bis auf die Abgabe der
scheinpflichtigen Übungsblätter) sausen zu lassen und nur aus Büchern zu
lernen. Manchmal, bei sehr schlechten Vorlesungen, ist dies sogar die einzige
sinnvolle Möglichkeit etwas zu lernen und seine Zeit sinnvoll zu nutzen. Hier
muss jedeR seinen/ihren eigenen Stil entwickeln. Sprecht Euch aber in jedem
Fall mit dem/der ÜbungsleiterIn ab, wenn ihr dort regelmäßig fehlen wollt, denn
manchmal ist die Anwesenheit und das unfreiwillige Vorrechnen notwendige
Voraussetzung für die Scheinvergabe.
