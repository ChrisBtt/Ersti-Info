\section{Arbeitsraum}

Seit dem letzten Semester gibt es am Philosophenweg 12 in den Räumen 60/61 einen studentischen 
Arbeitsraum, den wir vor allem für euch eingerichtet haben. Das Tolle daran ist, dass drei 
Tutor_inn_en eingestellt wurden, um Euch bei Fragen und Problemen zur Verfügung zu stehen. Das 
läuft sehr gut und wir haben viel positive Resonanz bekommen, 
besonders für die Tutor_inn_en. Das Konzept sieht vor, dass wir sowohl einen kleinen S
tillarbeitsbereich, aber auch einen Gruppenarbeitsraum für Diskussionen haben. Neben ein paar 
Standardbüchern gibt es dort auch eine Kaffeemaschine und einen Wasserkocher. In der 
Zukunft soll der Raum auch noch ausgebaut werden, allerdings muss er nun erst einmal über einen 
längeren Zeitraum angenommen werden, bevor noch mehr Geld dort hineingesteckt wird.
Besonders oft verliert man nämlich, im Übungszetteldschungel den Überblick, was wirklich 
wichtig ist. Genau dabei können einem
aber oft ältere Studis helfen. Auch sollen sie dir helfen deine Übungszettel besser zu lösen, 
indem sie hilfreiche Tipps geben. 

Um Rückmeldung dazu zu geben, könnt ihr einfach kurz die Umfrage unter 
\footnote{\url{http://goo.gl/forms/iDhmpx8kHx}} ausfüllen. Das hilft uns bei den 
Verhandlungen mit der Fakultät enorm und außerdem können wir so abschätzen, 
was die größten Probleme sind.

Der Raum ist während der Vorlesungszeit immer Montags bis Donnerstags von 14 bis 19 Uhr und 
Freitags bis 17 Uhr geöffnet. Falls dir also das KIP zu voll ist, du eh eine Übungsgruppe am 
Philosophenweg hast oder einfach am Philosophenweg Lust auf Kaffee hast, komm doch einfach mal vorbei.
