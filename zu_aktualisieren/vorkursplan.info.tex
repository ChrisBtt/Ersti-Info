% !TEX ROOT = ../ersti.tex
\section{Vorkurs: Informatik}
\label{vkinfo}
Da in den ersten beiden Semestern verhältnismäßig viel Mathe gemacht werden muss, ist der Mathe-Vorkurs auch allen zukünftigen Infostudierenden zu empfehlen. Er widmet sich einer Einführung in die universitäre Mathematik, sowie der Vorstellung einiger anderer Themengebiete. Siehe dazu weiter
\ifthenelse{\pageref{vkmathe} < \pageref{vkinfo}}{oben}{unten}.
Parallel zu den Mathevorträgen zum Thema „Gruppen“ und „Was ist Mathe?“ finden für die InformatikerInnen Informatik-Vorträge statt.

\vskip-5\parskip

\subsection{Programmierkurs}
Der Programmierkurs richtet sich an alle diejenigen von Euch, die noch keinerlei Kenntnisse im Bereich des Programmierens haben und sich im allgemeinen Umgang mit dem Computer unsicher fühlen. Der Kurs findet vom 21. bis 25. September im OMZ, \gls{INF} 350, statt und ist auch für MathematikerInnen zu empfehlen.

Die Kenntnisse werden in Informatik 1 (erstes Semester) bzw. spätestens in
Numerik 0 hilfreich sein. Davon abgesehen ist es nützlich erste Erfahrungen mit
unix-artigen Betriebssystemen („Linux“) zu machen, da sie im
naturwissenschaftlichen Bereich weit verbreitet sind.

Für den Programmierkurs ist aufgrund der begrenzten Plätze im Computer-Pool
eine vorherige Anmeldung erforderlich!

