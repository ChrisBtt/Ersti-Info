% !TEX ROOT = ../ersti.tex
%\setcounter{section}{0}%horizontale linie unterdrücken
\subsection*{\Large Vorkursplan: Physik}
\addcontentsline{toc}{section}{2.3\quad Vorkursplan: Physik}
\addtocounter{section}{1}
%boah ist das kompliziert diese blöde linie zu unterdrücken … (wahrscheinlich war ich zu blöd den einfachen weg zu sehen)
Der Vorkurs der Physik besteht aus einem Mathematik-Vorkurs der Fakultät für
Physik, gelesen von \dozentvorkurs. Seit Einführung des Bachelor wird auch ein
Kurs zu Schlüsselkompetenzen angeboten, der zum Bachelor-Programm gehört,
jedoch nicht verpflichtend ist. Ihr könnt also bereits durch den Vorkurs eure
ersten Credit-Points erhalten. Treffpunkt ist das Foyer des Hörsaalgebäudes
Physik, \Gls{INF} 308, um 9 Uhr. Die genauen Inhalte des Vorkurses legt der
Dozent je nach Kenntnisstand der Hörerschaft fest, wobei das zugehörige
Skript\footnote{\url{http://www.thphys.uni-heidelberg.de/~hefft/vk1/}} einen
guten Anhaltspunkt bietet.

Das Rahmenprogramm, welches ihr (ab der zweiten Woche) gemeinsam mit den Mathematikern und Informatikern habt, wird von der Fachschaft MathPhys organisiert und durchgeführt.% In der ersten Woche findet ein Spiel statt, bei dem ihr die Stadt kennenlernen könnt. mit "der zweiten Woche" ist die 1. Woche gemeint!

Den Vorkurs gibt es inzwischen auch als gebundenes Buch (siehe Buchliste). Geht ruhig mal in der ersten Vorkurswoche in die Universitätsbibliothek und leiht es euch sozusagen als Begleitbuch aus. (Vom Kauf möchten wir dennoch abraten.)

\begin{stundenplan}{Vorkursplan 0. Woche}{9}{20}
  \days{Mo 21.09.}{Di 22.09.}{Mi 23.09.}{Do 24.09.}{Fr 25.09.}

  \event{1}{09}{11}{Begrüßung}{INF 227}{xx}
  \event{1}{11}{13}{Campusführung}{INF 227}{xx}
  \event{1}{14}{18}{Vorkurs}{INF 227 / INF 306}{xxx}

  \event{2}{09}{13}{Vorkurs}{INF 227 / INF 306}{xxx}
  \event{2}{14}{18}{Vorkurs}{INF 227 / INF 306}{xxx}

  \event{3}{09}{13}{Vorkurs}{INF 227 / INF 306}{xxx}
  \event{3}{14}{18}{Vorkurs}{INF 227 / INF 306}{xxx}
  \event{3}{18}{99}{Fachschafts\-sitzung}{INF 305, R. 045}{x}

  \event{4}{09}{13}{Vorkurs}{INF 227 / INF 306}{xxx}
  \event{4}{14}{18}{Appel, Ei und Bier}{Exerzierplatz}{xxxx}

  \event{5}{09}{13}{Vorkurs}{INF 227 / INF 306}{xxx}
  \event{5}{14}{18}{Vorkurs}{INF 227 / INF 306}{xxx}
\end{stundenplan}

\begin{stundenplan}{Vorkursplan 1. Woche}{9}{22}
  \days{Mo 28.09.}{Di 29.09.}{Mi 30.09.}{Do 01.10.}{Fr 02.10.}

  \event{1}{09}{13}{Vorkurs}{INF 227 / Philweg 12}{xxx}
  \event{1}{14}{18}{Vorkurs}{INF 227 / Philweg 12}{xxx}
  \event{1}{20}{99}{Kneipentour}{Treffpunkt: Uniplatz}{xxxx}

  \event{2}{09}{13}{Vorkurs}{INF 227 / Philweg 12}{xxx}
  \event{2}{14}{18}{Vorkurs}{INF 227 / Philweg 12}{xxx}
  \event{2}{20}{99}{Spieleabend}{INF 288}{xxxx}

  \event{3}{09}{13}{Vorkurs}{INF 227 / Philweg 12}{xxx}
  \event{3}{14}{18}{Workshops}{Anmeldung: MÜSLI}{xxxx}
  \event{3}{18}{99}{Fachschafts\-sitzung}{INF 305, R. 045}{x}

  \event{4}{09}{13}{Vorkurs}{INF 227 / Philweg 12}{xxx}
  \event{4}{14}{18}{Vorkurs}{INF 227 / Philweg 12}{xxx}

  \event{5}{09}{13}{Vorkurs}{INF 227 / Philweg 12}{xxx}
  \event{5}{14}{18}{Vorkurs}{INF 227 / Philweg 12}{xxx}
\end{stundenplan}

\begin{stundenplan}{Vorkursplan 2. Woche}{9}{22}
  \days{Mo 05.10.}{Di 06.10.}{Mi 07.10}{Do 08.10}{Fr 09.10}

  \event{1}{09}{13}{Vorkurs}{INF 227 / Philweg 12}{xxx}
  \event{1}{14}{15}{Ausland}{}{xx}
  \event{1}{15}{20}{Workshops}{Anmeldung: MÜSLI}{xxxx}
  \event{1}{20}{99}{Kneipentour}{Treffpunkt: Uniplatz}{xxxx}

  \event{2}{09}{13}{Vorkurs}{INF 227 / Philweg 12}{xxx}
  \event{2}{14}{18}{Basiskurs}{Philweg 12}{xxx}
  \event{2}{20}{99}{Spieleabend}{INF 288}{xxxx}

  \event{3}{09}{13}{Vorkurs}{INF 227 / Philweg 12}{xxx}
  \event{3}{14}{18}{Basiskurs}{Philweg 12}{xxx}
  \event{3}{18}{19}{WasFachschaft?}{}{x}
  \event{3}{19}{99}{Fachschafts-sitzung}{INF 305, R. 045}{x}

  \event{4}{09}{13}{Vorkurs}{INF 227 / Philweg 12}{xxx}
  \event{4}{14}{18}{Basiskurs}{Philweg 12}{xxx}

  \event{5}{09}{11}{Vorkurs}{INF 227 / Philweg 12}{xxx}
  \event{5}{11}{13}{Dozentencafé}{}{xxxx}
  \event{5}{17}{20}{Aufbau Anfifete}{INF 227}{xxxx}
  \event{5}{20}{99}{Anfifete}{INF 227}{xxxx}
\end{stundenplan}

\noindent Beachtet, dass die Pläne auf dem Stand des Redaktionsschlusses sind. \textbf{Schaut auf der FS-Homepage\footnote{\url{http://mathphys.info/vorkurs/plan\#physik}} nach}, ob sich etwas geändert hat.

\subsection{Mentorenprogramm}
Um den Einstieg ins Studium zu erleichtern und euch Erstis eine weitere Möglichkeit zu bieten, sich auszutauschen, organisiert die Gleichstellungskommission der Fakultät ein Mentorenprogramm.

Die Mentorinnen und Mentoren sollen euch beratend begleiten und ihre Erfahrungen weitergeben. Es sind Studierende höherer Semester, DoktorandInnen, Postdocs und DozentInnen aller Fachbereiche der Physik und Astronomie. Die meisten Mentoren treffen sich in eurem ersten Semester ein paar mal mit ihren Gruppen in einem Café oder einer Kneipe. Vom Austausch über Freizeit und Studium bis zur Hilfe bei organisatorischen Belangen – den Ablauf sprechen die jeweiligen Mentoren mit ihren Mentees ab.

Wenn ihr Interesse oder Fragen zum Mentoringprogramm habt, schickt einfach eine Email an \url{gleichstellung@lsw.uni-heidelberg.de}. 
