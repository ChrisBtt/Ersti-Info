% !TEX ROOT = ../ersti.tex
% hier werden die Posten definiert, damit bei Neuwahlen nicht der
% Text durchsucht werden muss

% BITTE BEACHTEN:
%  - Titel sind Böse. Keine Titel. Keine Titel.
%  - Keine abschließenden Leerzeichen. Das ist einfach falsch und flahsc.

%Physik
\newcommand{\dekanphysik}{Hebecker}
\newcommand{\dekanphysiklang}{Arthur Hebecker}
\newcommand{\dekanphysikfoto}{bilder/hebecker_mon.jpg}
\newcommand{\prodekanphysik}{Hans-Christian Schulz-Coulon und Matthias Weidemüller}
\newcommand{\prodekanphysikA}{H.-C. Schulz-Coulon}
\newcommand{\prodekanphysikfotoA}{bilder/hcsc_mon.jpg}
\newcommand{\prodekanphysikB}{M. Weidemüller}
\newcommand{\prodekanphysikfotoB}{bilder/weidemueller_mon.jpg}
\newcommand{\studiendekanphysik}{Norbert Herrmann}
\newcommand{\studiendekanphysikfoto}{bilder/herrmann_mon.jpg}
\newcommand{\pruefausschussvorsitzphysik}{Cornelis Dullemond}
\newcommand{\studienberatungphysik}{Ulrich Uwer und Michael Hausmann}
\newcommand{\dozentvorkurs}{Herrn Thommes und Herrn Komnick}
\newcommand{\bafogphysik}{Norbert Herrmann}
\newcommand{\dekanatphysik}{Dewald-Klussmann}
\newcommand{\dekanatphysiktelefon}{+49\,62\,21 / 54\,-\,92\,98}
\newcommand{\frauenbeauftragtephysik}{Monica Dunford}
\newcommand{\pruefsekphysik}{Frau Hiemenz und Frau Nerger}
\newcommand{\pruefsekphysikfotoA}{bilder/hiemenz_mon.jpg}
\newcommand{\pruefsekphysikA}{Frau Hiemenz}
\newcommand{\pruefsekphysikfotoB}{bilder/nerger_mon.jpg}
\newcommand{\pruefsekphysikB}{Frau Nerger}


%Mathe
\newcommand{\dekanmathe}{Gertz}
\newcommand{\dekanmathelang}{Michael Gertz}
\newcommand{\dekanmathefoto}{bilder/gertz_mon.jpg}
\newcommand{\prodekanmathe}{Kay Wingberg}
\newcommand{\prodekanmathefoto}{bilder/wingberg_mon.jpg}
\newcommand{\studiendekanmathe}{Guido Kanschat}
\newcommand{\studiendekanmathefoto}{bilder/kanschat_mon.jpg}
\newcommand{\pruefausschussvorsitzmathe}{Gebhard Böckle}
\newcommand{\studienberatungmathe}{Karl Oelschläger und Winfried Kohnen}
\newcommand{\frauenbeauftragtemathe}{Anna Marciniak-Czochra}
\newcommand{\bafogmathe}{Markus Banagl}
\newcommand{\dekanatmathe}{Frau Häukäufer}
\newcommand{\dekanatmathetelefon}{+49\,62\,21 / 54\,-\,57\,58}
\newcommand{\pruefsekmathe}{Frau Kiesel}
\newcommand{\pruefsekmathefoto}{bilder/kiesel_mon.jpg}

%Info
\newcommand{\studiendekaninformatik}{Artur Andrzejak}
\newcommand{\studiendekaninformatikfoto}{bilder/andrzejak_mon.jpg}
\newcommand{\studienberatunginformatik}{Wolfgang Merkle}
\newcommand{\pruefausschussvorsitzinformatik}{Gerhard Reinelt}
\newcommand{\bafoginformatik}{Peter Bastian}
\newcommand{\pruefsekinfo}{Frau Sopka}
\newcommand{\pruefsekinfofoto}{bilder/sopka_mon.jpg}

%Uni
\newcommand{\frauenbeauftragteuni}{Jadranka Gvozdanovic}
\newcommand{\rektor}{Bernhard Eitel}
\newcommand{\fskstudisimsenat}{keinen der vier}

%Rest
\newcommand{\redaktionsschluss}{08.08.2015}
\newcommand{\volley}{im Dezember}                 % "Der nächste Termin ist ...   ."
\newcommand{\anfi}{9.\,Oktober 2015}               % Termin für die AnfiFete
\newcommand{\fsraum}{\Gls{INF} 305, Raum 045}
\newcommand{\auflage}{850}                        % wie viel Erstiinfos sollen
                                                  % gedruckt werden
\newcommand{\semester}{Wintersemester 2015/16}    % bislang nur fuer titel.tex

% wo fehlt noch
\newcommand{\fswe}{6. bis 8. November} % Dieses Semester findet es am <Datum> statt.

<<<<<<< HEAD
\newcommand{\mathphysromtermin}{16.\,Oktober 2016}
=======
\newcommand{\mathphystheotermin}{17.\,Oktober 2014}
>>>>>>> 328788f49d7a7431203fc2f02df428a4802f2cde

% diverse Minister

\newcommand{\wissenschaftsministerbawue}{Theresia Bauer (Grüne)}
\newcommand{\wissenschaftsministerbund}{Johanna Wanka (CDU)}


% Geldbeträge
%\newcommand{\studiengebuehren}{500}
\newcommand{\verwaltungsbetrag}{60}
\newcommand{\studentenwerksbeitrag}{74,80}
\newcommand{\vsbeitrag}{7,50}
\newcommand{\beitragssumme}{142,30}
\newcommand{\quasimi}{280}

% http://www.vrn.de/vrn/tickets/zeitkarten/studenten/vrn-semesterticket/index.html
\newcommand{\semesterticket}{150}

%%% Local Variables:
%%% mode: latex
%%% TeX-master: "ersti"
%%% End:
