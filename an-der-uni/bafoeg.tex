%\section{Die kleine BAföG-Liste}
\mathphyssecnobar{Die kleine BAföG-Liste}%FIXME

\paragraph{Soll ich BAföG beantragen?}
JA! Ob du anspruchsberechtigt bist, lässt sich nur schätzen. Vom BAföG sind 50\% Zuschuss, die andere Hälfte zinsloses Darlehen. Zurückzuzahlen muss man es  frühestens drei Jahre nach Studienabschluss, aber nur ab einer bestimmten Einkommenshöhe. Rückzahlungsaufschub ist jederzeit ohne Nachteile möglich!

\paragraph{Wie viel gibt es und wie lange und ab wann?}
Maximal \EUR{648} (ohne Kind, abhängig von Miet(e)verhältnis, Einkommen der Eltern, eigenem Vermögen, Krankenversicherung, Geschwistern, etc.)
Die Förderungsdauer richtet sich nach der Regelstudienzeit (Bachelor und konsekutiver Master = 6 + 4, aber es gibt fachspezifische Ausnahmen), Anspruchszeit zählt ab Immatrikulation und ist nicht aufsparbar! Die Auszahlung geschieht in der Regel erst, wenn Bescheid eingegangen ist, bei dringender Bedürftigkeit ist  ggf. Vorschuss (in voller BAföG-Höhe, der hinterher verrechnet wird) möglich.

\paragraph{Wo und bis wann muss ich den Antrag stellen?}
Das geht beim BAföG-Amt des Studentenwerks im Marstallhof 3 je Mo.\ -- Fr.\ 8 – 18 Uhr. Kurzberatung im Neuenheimer Feld gibt es Mo.\ – Do.\ 10.00 -- 17.00 Uhr und Fr.\ 10.00 -- 15.00 Uhr. Das macht man am besten sobald man die Immatrikulationsbescheinigung hat, denn BAföG gibt es erst vom Antragsmonat an, wobei der frühestmöglichste Termin der Semesteranfang ist.

\paragraph{Was muss ich beim Antrag beachten?}
Antrag so früh wie möglich stellen, weil es teilweise zu langen Bearbeitungszeiten kommen kann. Der Antrag sollte idealerweise vollständig sein, da so Rückfragen entfallen und der Antrag schneller bearbeitet werden kann. Ansonsten gilt: lieber schnell beantragen und Papiere nachreichen. Die Angaben sollten wahrheitsgemäß sein, denn das Amt kann die Finanzdaten prüfen.

\paragraph{BAföG und nun? Was sonst noch zu beachten wäre.}
Man muss nach dem vierten Fachsemester seine Studienleistungen bescheinigen lassen!
Familiäre und persönliche wirtschaftliche Veränderungen immer sofort melden (Schwester/Bruder in Ausbildung/Arbeit, Arbeitslosigkeit, etc.)
Es gibt auch Auslands-BAföG und viele sonstige Ausnahmen, zu denen man sich am besten beraten lässt! Es kann sich lohnen!
Zuverdienst bis maximal \EUR{4\,880} Brutto im Jahr oder \EUR{450} im Monat (mit vielen Ausnahmen).\\[5mm]

\noindent Abschließend: Nutzt die Sprechstunde vom Sozialreferat des \gls{StuRa} (Mi.\ und Fr.\ 11 -- 13 Uhr). Die helfen Euch beim Antrag, erklären Ausnahmen, geben Tipps (um spätere Probleme zu vermeiden, bei denen sie natürlich auch helfen).
